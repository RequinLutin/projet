\documentclass[a4paper,11pt]{paper}

%====================== PACKAGES ======================

\usepackage[french]{babel}
\usepackage[utf8x]{inputenc}
\usepackage[T1]{fontenc}
\usepackage{amsmath,graphicx,url,hyperref,array,tabularx}
%police et mise en page (marges) du document
\usepackage[top=2cm, bottom=2cm, left=2cm, right=2cm]{geometry}

\usepackage[usenames,dvipsnames]{color}
\usepackage[listings]{tcolorbox}% http://ctan.org/pkg/tcolorbox
\definecolor{rougemeca}{RGB}{203,10,32}

\usepackage{tikz}


%====================== INFORMATION ET REGLES ======================

\author{
        Premier AUTEUR\\
        Deuxième AUTEUR\\
        Troisième AUTEUR
}
\title{
	Rapport de projet technique\\
        Quelques consignes
}
\newcommand{\annee}{2021-2022}
\newcommand{\typeprojet}{\color{white}{HAY507Y - CalNumMec}}
\newcommand{\scolarite}{CUPGE L3}
\newcommand{\encadrant}{Premier ENCADRANT}
\newcommand{\client}{Société}
%
\addto\captionsfrench{%
  \renewcommand{\listfigurename}{Liste des Figures}%
  \renewcommand{\listtablename}{Liste des Tableaux}%
}


%rajouter les numérotation pour les \paragraphe et \subparagraphe
\setcounter{secnumdepth}{4}
\setcounter{tocdepth}{4}


%======================== DEBUT DU DOCUMENT ========================

\begin{document}
\sffamily

%!TEX root = ./Consignes_Rapport_Technique.tex

% page de garde
\begin{titlepage}

        \hspace*{-2.7cm}\rotatebox[origin=rB]{90}{%
        \tcbox[tcbox raise base,colback=rougemeca!75!white,boxrule=0pt,arc=0pt]%
        {
        \parbox[c][3cm]{24.2cm}{%
        \begin{center}%
        \sffamily\bfseries\Huge%
        \ \\[20pt] \typeprojet ~--- \scolarite \\ \ \\
        \end{center}%
        }%
        }}
        \vspace*{-25.7cm}

       \hspace*{1.5cm}\raggedleft
       \begin{minipage}{0.95\textwidth}
        \begin{center}
        % Upper part of the page. The '~' is needed because only works if a paragraph has started.
	\includegraphics[height=2.6cm]{./figures/logo_um.png}\hfill
        \includegraphics[height=3cm]{./figures/logo_dpt.png}\hfill
        \raisebox{-0.0cm}{\includegraphics[height=2.7 cm]{./figures/logo_FDS.png}}~\\[1cm]
        \textbf{\LARGE Département de Mécanique\\[0.25cm]
        Faculté des Sciences  --- Université de Montpellier}\\[1.5cm]
        \textsc{\Large }\\[0.5cm]
        % Title
        \rule{\linewidth}{0.5mm} \\[0.4cm]
        {\huge \bfseries \makeatletter\@title\makeatother \\[0.4cm] }
        \rule{\linewidth}{0.5mm} \\ \begin{minipage}{\textwidth}\raggedleft {\Large \color{Gray} année \annee}\end{minipage} \ \\[2.5cm]
        % Author and supervisor
        \begin{minipage}{0.48\textwidth}
        \begin{flushleft} \large
        {\color{Gray} Auteurs}\\
        \makeatletter\@author\makeatother
        \end{flushleft}
        \end{minipage}
        \begin{minipage}{0.48\textwidth}
        \begin{flushright} \large
        {\color{Gray} Client} \\
        \client\\
        {\color{Gray} Encadrant} \\
        \encadrant
        \end{flushright}
        \end{minipage}
        
        \vspace*{8.9cm}
        % Bottom of the page
        {\large \today}
        \end{center}
        \end{minipage}
        
\end{titlepage}

%page blanche
\newpage
~
%ne pas numéroter cette page
\thispagestyle{empty}
\newpage
 % insérer la page de garde

%\renewcommand{\abstractnamefont}{\normalfont\Large\bfseries}
%\renewcommand{\abstracttextfont}{\normalfont\Huge}

\thispagestyle{empty}
        \begin{abstract}
        \hskip7mm        
        Le résumé doit être concis (10-15 lignes), clair et doit permettre simultanément: (1) au non spécialiste de comprendre la thématique et les enjeux et (2) au spécialiste de savoir s'il trouvera l'information qu'il recherche. En général, il ne doit pas contenir acronyme, de sigles ou de notations qui nécessiteraient un pré-requis. Il est souvent judicieux d'en proposer un version en anglais.          
        \end{abstract}

\newpage

% table des matières
\tableofcontents
\thispagestyle{empty}
%ne pas numéroter le sommaire
\newpage
\thispagestyle{empty}
% table des matières
\listoffigures
\listoftables
\clearpage
\newpage

%espacement entre les lignes d'un tableau
%\renewcommand{\arraystretch}{1.5}

%====================== CORPS DU TEXTE ======================

%recommencer la numérotation des pages à "1"
\setcounter{page}{1}

\section{Rédaction du texte}

Le texte doit être rédigé dans un français convenable (correcteur orthographique a minima), en respectant les règles typographiques d'usage, sur la ponctuation notamment\footnote{voir par exemple \href{http://revues.refer.org/telechargement/fiche-typographie.pdf}{\texttt http://revues.refer.org/telechargement/fiche-typographie.pdf}}. \\

Le texte est rédigé à la \emph{voie passive}, en limitant les personnifications. On écrira: "le travail réalisé concerne la mécanique", plutôt que "notre travail concerne la mécanique". On écrira: "l'hypothèse des petites perturbations a été formulée", plutôt que "nous avons fait l'hypothèse des petites perturbations". Ce n'est pas "notre problème", ni "notre travail", ni "notre méthode"; il s'agit du "problème considéré", du "travail réalisé", de la "méthode proposée".  \\

Autant que faire ce peut, un rapport scientifique doit être autosuffisant: toutes les données utilisées doivent être listées, ainsi que les hypothèses formulées, les équations considérées doivent être rappelées en précisant toutes les notations ou les sigles (au fil du texte ou dans une nommenclature), les annexes peuvent permettre d'indiquer un rappel de contexte que le lecteur est sensé connaître, les références sont complètes et détaillées. Un rapport a vocation à être lu/utilisé par des personnes n’ayant pas travaillé sur le sujet.

Les passages très techniques qui peuvent nuire à la lecture sont préférentiellement mis en annexe. 

Les scripts informatiques, les morceaux de programmes, etc, n'ont pas vocation à apparaître explicitement ; il est préférable d'expliquer les algorithmes et les méthodes utilisés.\\

Le texte doit contenir une introduction et une conclusion. L’introduction doit donner le cadre de l’étude en positionnant le problème d’un point de vue scientifique, industriel, sociétal, économique. La conclusion doit constituer plus qu’une simple redite de ce qui est déjà écrit dans le corps du texte; elle doit élargir la discussion, insister sur les limites du travail réalisé, la (non) pertinence des hypothèses, les manières de l’étendre, etc. Le lecteur pressé doit pouvoir se contenter du résumé, de l'introduction et de la conclusion. 

\section{Les tables et figures}

Les tables et les figures sont insérées dans le texte \emph{uniquement} si elles sont informatives; il ne s'agit pas de décorer le texte, ni de le meubler. Les figures ne sont pas de simples copies d’écran: il faut réaliser des figures spécifiques, avec des noms et des unités sur les axes, des tailles de labels adaptées. 
\\

Le tableau \ref{table:1} est un exemple d'un tableau référence par les commandes  \LaTeX.
 
\begin{table}[h!]
\centering
\begin{tabular}{||c c c c||} 
 \hline
 Col1 & Col2 & Col2 & Col3 \\ [0.5ex] 
 \hline\hline
 1 & 6 & 87837 & 787 \\ 
 2 & 7 & 78 & 5415 \\
 3 & 545 & 778 & 7507 \\
 4 & 545 & 18744 & 7560 \\
 5 & 88 & 788 & 6344 \\ [1ex] 
 \hline
\end{tabular}
\caption{Table to test captions and labels}
\label{table:1}
\end{table}


des informations complémentaires sont disponibles sur : https://fr.sharelatex.com/learn/Tables

Les figures sont numérotées, légendées et obligatoirement citées une fois dans le texte par un renvoi à leur numérotation (on évitera "la figure ci-dessous" qui n'est pas robuste au changement de pagination, on préfèrera "la \figurename~\ref{fig:logo}"). La légende peut être comprise sans devoir lire le corps du texte de manière approfondie.

\begin{figure}[h]
\centering
\includegraphics[height=2.5cm]{./figures/logo_FDS.png}
\caption{Logo de la Faculté des sciecnes en janvier 2018.}
\label{fig:logo}
\end{figure}
 
\section{Les équations}

Les équations doivent être numérotées, a minima lorsqu'elles ont besoin d'être référencées. De nouveau, on évitera "l'équation qui suit" et on préfèrera "l'équation~\eqref{eq1}" pour rester robuste à la modification de pagination.
\begin{eqnarray}
div(\sigma)=0
\label{eq1}
\end{eqnarray}
On précisera l'intégralité des notations, par exemple: dans l'équation~\eqref{eq1} le symbole $div$ désigne la divergence spatiale et $\sigma$ est le tenseur des contraintes de Cauchy.

\section{Les références}

Un travail scientifique et technique s'appuie sur des références solides issues d'une littérature validée par la communauté (par exemple, un ouvrages de référence \cite{maxwell_1873}, un mémoire de thèse, un chapitre des sciences et techniques de l'ingénieur ou d'une collection \cite{Moreau1988}, un rapport technique \cite{Karb_et_al_KFK_1983}, un article scientifique tiré de revue avec comité de lecture \cite{grange_et_al_IJF_2000}, etc.). Les références à des pages internet doivent être une exception (même wikipedia).  

La liste bibliographique doit être suffisamment précise: un lecteur doit pouvoir trouver ou commander chaque référence en quelques minutes. Une référence "cours de mécanique des solides Polytech Montpellier” n'est pas suffisante.

L'utilisation de tout ou partie d'un ouvrage antérieur à la rédaction du présent rapport doit faire l'object d'une citation. Le manquement à cette règle constitue un plagiat dont les conséquences peuvent être lourdes (en janvier 2018, jusqu’à deux ans d’emprisonnement et 150 000 euros d’amende). L'Université est doté d'un système de détection des plagiats.

Les "archives ouvertes" sur internet (cours, rapports, thèses, HAL, etc.) sont protégées par le code de la propriété intellectuelle. 


%%%%%%%%%%%%%%%%%%%% LA BIBLIOGRAPHIE
\newpage
\section{Bibliographie}
\nocite{*}
\bibliographystyle{plain}
\bibliography{bibliographie.bib}
%%%%%% Annexe spécifique au PIM
\newpage
%!TEX root = ./Consignes_Rapport_Technique.tex

\begin{appendix}
\section{Annexe spécifique a HLM509}

\subsection{Contours des projets}

\subsubsection{Objectifs}
Les objectifs principaux des PIM sont:
\begin{itemize}
\item Un premier contact avec la démarche de modélisation qui est de plus en plus présente dans les bureaux d’études \\ \ \\
\begin{tikzpicture}

\def \radius {3cm}
\def \margin {30} % margin in angles, depends on the radius
  \node[draw, rectangle] at ({360/4 * (1 - 1)}:\radius) {\begin{tabular}{c}situation\\physique\end{tabular}};
  \draw[->, >=latex] ({360/4 * (1 - 1)+\margin}:\radius) 
    arc ({360/4 * (1 - 1)+\margin}:{360/4 * (1)-\margin}:\radius);
%
  \node[draw, rectangle] at ({360/4 * (2 - 1)}:\radius) {\begin{tabular}{c}modèle\\physique\end{tabular}};
  \draw[->, >=latex] ({360/4 * (2 - 1)+\margin}:\radius) 
    arc ({360/4 * (2 - 1)+\margin}:{360/4 * (2)-\margin}:\radius);
%
  \node[draw, rectangle] at ({360/4 * (3 - 1)}:\radius) {\begin{tabular}{c}formalisme\\mathématique\end{tabular}};
  \draw[->, >=latex] ({360/4 * (3 - 1)+\margin}:\radius) 
    arc ({360/4 * (3 - 1)+\margin}:{360/4 * (3)-\margin}:\radius);
%
  \node[draw, rectangle] at ({360/4 * (4 - 1)}:\radius) {\begin{tabular}{c}résolution\\analytique\\numérique\end{tabular}};
  \draw[->, >=latex] ({360/4 * (4 - 1)+\margin}:\radius) 
    arc ({360/4 * (4 - 1)+\margin}:{360/4 * (4)-\margin}:\radius);
%
\end{tikzpicture}

\item Apprendre différemment
\end{itemize}

\subsubsection{Règles du jeu}
Pour l'essentiel les Projets numériques consistent en:
\begin{itemize}
\item 1 groupe + 1 tuteur + 1 sujet
\item un sujet ouvert
\item un tuteur expert
\item des heures de consultation
\item un point de vue opposé d’un cours magistral: il faut demander pour recevoir
\end{itemize}

\subsubsection{Quelques dates}
\begin{itemize}
\item Début semestre d'année : constitution des groupes et attribution des sujets
\item Mi Semestre : remise du premier document et présentations des premiers résultats 
\item Fin du semestre : remise du rapport (15-20 pages) et soutenance orale
\end{itemize}


\subsubsection{Quelques pièges à éviter}
\begin{itemize}
\item travail d'équipe
\item gestion du temps ... et du stress 
\item le temps des notes faciles est révolu
\item Notation mixte globale/individuelle
	\begin{itemize}
	\item 1/3 : rapport (double correction)
	\item 1/3 : présentation (jury)
	\item 1/3 : travail (tuteur) + soutenance de mi-parcours
	\end{itemize}
\end{itemize}


\subsubsection{Grille d'évaluation}
Les points suivants sont évalués en priorité:
\begin{itemize}
\item Description du contexte, motivation et objectif de l’étude
\item Présentation des hypothèses de travail, écriture du modèle physique
\item Compréhension de la méthode de résolution
\item Analyse critique/physique des résultats obtenus. Retour éventuel sur les hypothèses
\item Qualité de la forme: figure, orthographe, précision du langage
\end{itemize}

\subsection{Conseils pour la présentation orale des projets}

\subsubsection{Conseils généraux}
\begin{itemize}
\item La "star" c’est vous, pas l’écran
\item Soyez clair, quitte à simplifier votre discours en ne discutant pas tous les détails: pour être appréciée, une présentation doit d’abord être comprise par le public
\item Ne pas montrer ce que l’on ne présente pas. Le matériel est soit indispensable et on en parle, soit inutile et on ne le montre pas
\item Parler suffisamment fort, et regarder votre public
\item Numéroter les planches, respecter le temps imparti, répéter
\item Pour vérifier que votre support de présentation oral est adapté (taille des caractères, couleurs, contraste): allez en salle SC001, projeter vos slides et mettez vous au fond de la salle. Si vous ne voyez pas l’ensemble des informations écrites, ce n’est pas bon !
\end{itemize}

\subsubsection{Introduction}
\begin{itemize}
\item Partie cruciale de l’exposé: doit être comprise par tous. Entre 10 et 30\% de l’exposé pour bien motiver l’étude
\item Elle doit répondre aux questions suivantes :
\item Quel est le contexte industriel ? Scientifique ?
\item Quels sont les phénomènes mis en jeu dans le problème ?
\item Que faut-il étudier ? Pourquoi ? Quelles quantités intéressantes ?
\item Qu’est-ce que votre étude peut apporter ?
\item Mettre des photos, des films, des schémas : des images !
\end{itemize}

\subsubsection{Présentation du problème}
\begin{itemize}
\item Rappeler les hypothèses de travail
\item Mettre des schémas de la configuration étudiée
\item Ne pas surcharger d’équations
\item "Faire parler" les équations: expliciter les termes, dire à quelle physique ils correspondent. Lire uniquement les équations à haute voix ne sert à rien.
\end{itemize}

\subsubsection{Présentation des résultats - Figures}
\begin{itemize}
\item Mettre des axes avec des unités, des légendes
\item Mettre des traits épais, des couleurs vives (pas de jaune, vert clair etc)
\item Les légendes/labels/titres/traits doivent être \textbf{très} gros
\item Les figures des présentations ne sont pas les mêmes que ceux des rapports: \textbf{chaque figure doit être faite deux fois}
\item Comparaison deux 2 courbes : les mettre sur le même graphe ou tracer la différence
\item Comparaison de champs 2D: utiliser les mêmes échelles de couleur ou tracer la différence
\item Pour les problèmes instationnaires, les films sont à privilégier
\end{itemize}

\end{appendix}


\end{document}